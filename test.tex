\documentclass[openany][a4paper,11pt]{book} 
\usepackage{amsmath,amssymb}  
\usepackage{latexsym}  
\usepackage{CJKutf8} 
\usepackage{geometry}
\usepackage{setspace}
\usepackage{titlesec}
\usepackage{fancyhdr}
% \renewcommand{\CJKglue}{\hskip 1pt plus 0.08\baselineskip}
% \renewcommand{\CJKglue}{\hskip 1pt} % 设置字间距
\setlength{\parskip}{0.5\baselineskip} % 段间距
\geometry{left=3.5cm,right=3cm,top=2.5cm,bottom=2.5cm}    % 页边距
% \renewcommand\thesection{\arabic{section}~}   % 重定义章节编号
\renewcommand\thesection{\arabic{section}~}
% \titleformat*{\section}{\centering\Huge\bfseries}     % 设置 section 文字居中
\begin{document}  
\begin{spacing}{1.2}        % 行间距
\renewcommand\arraystretch{1.5} % 表格行间距
\begin{CJK}{UTF8}{gbsn}     % 设置字体
% 声明一个新命令,用于表格内强制换行
\newcommand{\tabincell}[2]{\begin{tabular}{@{}#1@{}}#2\end{tabular}}
\pagestyle{empty}               
% 设置页类型为:页眉页脚为空,参数还有plain/headings/myheadings
\begin{Huge}
\title{\bf MongoDB and Python}      % 设置标题页,即首页
\end{Huge}
\author{书籍作者:Niall O’Higgins \\ 笔记:王斌}
\date{2013年2月}

\maketitle                      % 设置标题页
\tableofcontents                % 生成目录
\thispagestyle{empty}           % 不显示页码
\newpage
\pagestyle{plain}
\setcounter{page}{1}
\newpage
\section{Connecting to MongoDB with Python(用Python连接MongoDB)}

为了避免输入数据类型的错误,强烈建议在编写的程序中进行校验数据类型。

\newpage
\section{Update-or-Insert: Upserts in MongoDB(Upserts使用方法)}

可以使用Upserts的方法有三种:

\begin{enumerate}
\def\labelenumi{\arabic{enumi}.}
\item
  Collection.save()

  save()与insert()的用法非常相似,不同之处在于save()可以同时执行更新插入,但在同一个命令中不能插入多个文档或记录。

  另外,save()只能应用于'\_id'的存在检测,对于其它字段则不适用。
\item
  Collection.update()

  update()可以检测其它字段的是否存在。如果存在则执行更新操作,如果不存在则执行插入操作。
\item
  Collection.find\_and\_modify()

\begin{verbatim}
# 当要更新一条记录时,传统的做法是先查询,如果存在则更新,如果不存在则插入相关数据。(**不推荐**)
# Naive, bad implementation without upsert=True
def edit_or_add_session(description, session_id):
    # We must query first, 
    # becase we don't know whether this session_id already exists.
    # If we attempt to update a non-existent document, no write will occur.
    session_doc = dbh.sessions.find_one({"session_id":session_id})
    if session_doc:
更新操作,然后返回改变后的新值。
# $inc 是自动增量操作;$dec 是自动减量操作。
# User X adds $20 to his/her account, so we atomically increment
# account_balance and return the resulting document
ret = dbh.users.find_and_modify({"username":username},
    {"$inc":{"account_balance":20}}, safe=True, new=True)
new_account_balance = ret["account_balance"]
\end{verbatim}

\newpage
\section{Fast Accounting Pattern(快速计数模式)}
\clearpage      % 解决最后一页的页眉页脚格式BUG
\end{CJK} 
\end{spacing}
\end{document}

