\documentclass[12pt]{ctexart} 
%文档默认字号12磅,中文模式ctexart

%\documentclass[12pt]{article}
%\usepackage{zhfontcfg}
%archlinux系统使用,其中 zhfontcfg 宏包见 code 仓库。

\usepackage{geometry}
\geometry{a4paper,centering,scale=0.8}
%设置纸张宏包geometry,设置为A4纸幅面,版心居中,长宽占页面的0.8倍。
%\geometry{left=1cm,right=2cm,top=3cm,bottom=4cm}
%页边距为左1cm,右2cm,上3cm,下4cm
%\geometry{papersize={20cm,15cm}}
%纸张的长度设置为20cm,宽度设置为15cm

%在页眉左边写上我的名字,中间写上今天的日期,右边写上我的电话;页脚的正中写上页码;页眉和正文直接有一道宽为 0.4pt 的横线分割,可以在导言区加上如下几行。
%\usepackage{fancyhdr}
%\pagestyle{fancy}
%\lhead{\author}
%\chead{\date}
%\rhead{152xxxxxxxx}
%\lfoot{}
%\cfoot{\thepage}
%\rfoot{}
%\renewcommand{\headrulewidth}{0.4pt}
%\renewcommand{\headwidth}{\textwidth}
%\renewcommand{\footrulewidth}{0pt}

%\usepackage{indentfirst}
%\setlength{\parindent}{2.45em}
%首行缩进设置。2.45em是中文小四号字大小两个中文字的长度。

\usepackage{setspace}
\onehalfspacing
%将行距设置为1.5倍。

\addtolength{\parskip}{.4em}
%在原有基础上,增加段间距0.4em,如果要减小段间距,只需将该数值改为负值即可。

%\usepackage[nottoc]{tocbibind}
%增加目录宏包tocbibind,宏包默认会在目录中加入目录项本身、参考文献、索引等项目,这里使用nottoc选项取消了在目录中显示目录本身。

%\usepackage{graphicx}
%插入图片或图表的宏包。

%newenvironment{myquote}{\begin{quote}\kaishu\zihao{-5}}{\end{quote}}
%定义一个新环境myquote,这里设置字体为楷书,字号为-5
%使用:\begin{myquote} 正文 \end{myquote}

%\newcommand\degree{^\circ}
%新命令为\degree,用来代替原始命令^\circ,这是一个表示度数的符号。

\title{你好,world!}
\author{wb}
\date{\today}
%设置作者、标题、日期

\begin{document}
\maketitle
\tableofcontents
%字体设置
%\textrm{文字} \rmfamily
%罗马型字体
%\textsf{文字} \sffamily
%无衬线字体
%\texttt{文字} \ttfamily
%打字机字体
%\textmd{文字}
%\mdseries
%中等字体,系统默认系列
%\textbf{文字}
%\bfseries
%字体加宽加粗。
%\textup{文字}  \upshape
%直立效果
%\textit{文字} \itshape
%意大利效果
%\textsl{文字} \slshape
%倾斜效果
%\textsc{文字} \scshape
%小型大写效果
%\textnormal{文字} \normalfont
%普通字体,相当于\rmfamily\mdseries\upshape

%\phantom{参数} \hphantom \vphantom 表示水平方向和垂直方向的幻影。
%幻影空格,作用是产生与参数内容一样大小的空盒子,没有内容,就象是参数的一个幻影一样。
%\linebreak指定一行的断点,上一行按完整一行散开对齐。可以带一个从0到4的可选参数,表示允许断行的程度,0表示不允许断行,默认的4表示必须断行。
%\\ 直接另起一行,上一行保持原来的样子。

%插入图片
%\includegraphics[width=.8\textwidth]{a.jpg}
%图片宽度缩放到页面宽度的80%,图片高度会按比例缩放。

\emph 加重
$a+b$
$a^2+b^2=c$
$\angle A=\pi/2$
\end{document}
