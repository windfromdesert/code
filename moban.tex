\documentclass[12pt]{ctexart} 
%文档默认字号12磅,中文模式ctexart
\usepackage{geometry}
\geometry{a4paper,centering,scale=0.8}
%设置纸张宏包geometry,设置为A4纸幅面,版心居中,长宽占页面的0.8倍。
%\usepackage[nottoc]{tocbibind}
%增加目录宏包tocbibind,宏包默认会在目录中加入目录项本身、参考文献、索引等项目,这里使用nottoc选项取消了在目录中显示目录本身。
%newenvironment{myquote}{\begin{quote}\kaishu\zihao{-5}}{\end{quote}}
%定义一个新环境myquote,这里设置字体为楷书,字号为-5
%使用:\begin{myquote} 正文 \end{myquote}
%\newcommand\degree{^\circ}
%新命令为\degree,用来代替原始命令^\circ,这是一个表示度数的符号。
\begin{document}

%字体设置
%\textrm{文字} \rmfamily
%罗马型字体
%\textsf{文字} \sffamily
%无衬线字体
%\texttt{文字} \ttfamily
%打字机字体
%\textmd{文字}
%\mdseries
%中等字体,系统默认系列
%\textbf{文字}
%\bfseries
%字体加宽加粗。
%\textup{文字}  \upshape
%直立效果
%\textit{文字} \itshape
%意大利效果
%\textsl{文字} \slshape
%倾斜效果
%\textsc{文字} \scshape
%小型大写效果
%\textnormal{文字} \normalfont
%普通字体,相当于\rmfamily\mdseries\upshape

%\phantom{参数} \hphantom \vphantom 表示水平方向和垂直方向的幻影。
%幻影空格,作用是产生与参数内容一样大小的空盒子,没有内容,就象是参数的一个幻影一样。
%\linebreak指定一行的断点,上一行按完整一行散开对齐。可以带一个从0到4的可选参数,表示允许断行的程度,0表示不允许断行,默认的4表示必须断行。
%\\ 直接另起一行,上一行保持原来的样子。
test
\emph 加重

$a+b$

$a^2+b^2=c$

$\angle A=\pi/2$

中文
中文
\end{document}
